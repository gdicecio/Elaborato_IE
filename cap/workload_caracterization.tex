\chapter{Workload Caracterization}

\section{Descrizione Dataset}
Il dataset di partenza è composto da \textbf{3000 righe} e \textbf{24 colonne}, ciascuna delle quali rappresenta uno dei parametri del sistema oggetto di studio. Molto probabilmente si tratta di parametri caratterizzanti l'esecuzione di un processo (formato da + threads) su un sistema operativo (?) vedi se lasciare questa ipotesi.

\section{Filtraggio}

\subsection{Colonne Identiche}
Innanzitutto osservando il workload e plottandone le distribuzioni ci siamo resi conto della presenza di ben 4 colonne costanti:

\begin{itemize}
	\item \textbf{Active}
	\item \textbf{AnonPages}
	\item \textbf{AvbLatency}
	\item \textbf{Error}
\end{itemize}
Essendo tali non spiegano varianza, dunque possono essere tranquillamente trascurate.
Inoltre, osservando le distribuzioni dei parametri \textbf{WriteBack} e \textbf{MemFree} siamo giunti alla conclusione che le due colonne sono identiche e forniscono quindi la stessa informazione. Per questo motivo abbiamo deciso di trascurare una delle due, in particolare quella di WriteBack.
\\
A questo punto, dalle iniziali 24 colonne siamo riusciti ad ottenerne 19.

\subsection{Outlier}
Osservando l'andamento dei parametri:
\begin{itemize}
	\item \textbf{VmSize}, quanta memoria virtuale utilizza l'intero processo.
	\item \textbf{VmHWM}, di quanta RAM il processo necessita al massimo.
	\item \textbf{VmRSS}, quanta RAM il processo sta correntemente usando.
	\item \textbf{VmPTE}, quanta memoria Kernel è occupata dalle entries della tabella delle pagine.
\end{itemize}
si è notato che essi presentano un outlier isolato in comune associato alla prima riga del dataset. Analizzando gli altri parametri (\textit{MemFree}, \textit{Dirty}, \textit{PageTables}, \textit{Buffer}, ...) è stato possibile evidenziare che anche per la maggior parte di essi lo è, ma non è un punto isolato. 
\\
\\
L'ipotesi fatta è che con molta probabilità le prime righe del dataset (da 0 a 100 circa), rappresentano la fase di avvio del processo e l'outlier oggetto di studio è la prima istanza di questa fase. Dato che l'obiettivo della caratterizzazione del workload è quello di analizzare le prestazioni a regime del sistema oggetto di studio (in questo caso), si è deciso di trascurare quel singolo outlier. E' bene notare che in ogni caso le informazioni riguardo questa fase di avvio non saranno del tutto perse visto che non abbiamo eliminato tutti gli outlier (di tutti i parametri) associati a tale fase (vedi meglio sto fatto, ma secondo me è cosi).
\\
\\
Un secondo outlier che può essere agevolmente rimosso è la riga 512 in cui il parametro \textbf{Slab} assume valore 4. Oltre ad avvicinarsi molto al valore medio assunto da Slab (zero), esso risulta essere un outlier solo per il parametro stesso dato che per gli altri è un valore compreso tra i quartili. Una sua rimozione quindi non influenza gli indici di caratterizzazione sintetica dei parametri del workload complessivo.
\\
\\
Un terzo outlier è il valore 1760 del parametro \textit{Slab}, associato alla riga 90 del workload. Rispetto al precedente, tale outlier richiede un' analisi più approfondita visto che influenza significativamente l'andamento di parametri quali \textit{Mapped} e \textit{PageTables}. 
Lo Slab molto probabilmente si riferisce ad un particolare meccanismo di allocazione/deallocazione della memoria nel Kernel.
Dato che non sappiamo se questo è un evento che si verifica di norma a quel punto dell'esecuzione del processo (non so sincer) o è stato un caso isolato associato a questa misurazione, si è deciso di non eliminarlo. 
\newpage

\section{PCA}
Considerazioni sulla PCA, una sola PCA

\section{Clustering}
Vari tentativi con grafico della devianza persa
Cluster 15; Cluster 12; Cluster 10; Cluster 8; Cluster 5;

\section{Workload Sintetico}