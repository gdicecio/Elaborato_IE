\chapter{Workload Caracterization}

\section{Filtraggio}

\subsection{Oulier}
Nelle osservazioni della memoria virtuale dei processi (Vm) si è notato un oulier isolato in comune a suddetti paramenti del workload. Analizzando le altre osservazioni (MemFree, Dirty, WriteBack, Buffer, ...) si è notato che le prime righe rappresentano una fase di avvio del sistema, e l'outlier in considerazione fa parte proprio di questa fase. A tal proposito inoltre esso è pur sempre un outlier ma non isolato per gli altri paramentri. Dato che l'analisi del workload non si focalizza sulla condizione di funzionamento sopra citata, si è preferito togliere quel singolo outlier.

Un secondo outlier che può essere rimosso è la riga in cui il parametro \textit{Slab} assume valore 4. Esso infatti è un outlier solo per il parametro \textit{Slab}, mentre per gli altri parametri è un valore compreso tra i propri quartili. Una sua rimozione quindi non influenza gli indici di caratterizzazione sintetica dei parametri del workload complessivo.

Un terzo outlier è il valore 1760 del parametro \textit{Slab}. Come nel caso precedente tale punto si trova in media con tutti gli altri parametri tranne che per \textit{Mapped}

\subsection{Colonne Identiche}
\begin{itemize}
	\item Active
	\item AnonPages
	\item AvgLatency
	\item Error
\end{itemize}


\section{Principal Component Analysis - PCA}
Considerazioni sulla PCA, una sola PCA

\section{Clustering}
Vari tentativi con grafico della devianza persa
Cluster 15; Cluster 12; Cluster 10; Cluster 8; Cluster 5;

\section{Workload Sintetico}