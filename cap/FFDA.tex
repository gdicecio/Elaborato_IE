\chapter{FFDA}
Log-based FFDA di sistemi complessi a larga scala. I dati sono stati collezionati dai supercalcolatori Mercury e Blue-Gene. 
\\Breve descrizione dell'architettura dei due sistemi
\section{Analisi dei log}
Formato entries per entrambi.
\\RICORDA!! I dati sono stati già filtrati, ma comunque sono presenti ridondanze. Quindi necessitano una manipolazione.
\section{Manipolazione}
Fatta tramite tecniche di coalescenza, le quali consentono di raggruppare entries ricercando correlazioni sui dati. 
\\Coalescenza spaziale: raggruppiamo entries provenienti da + nodi diversi cercando di capire se si sono verificate nello stesso intervallo temporale specificato dalla finestra temporale W seleizonata.
\subsection{Scelta W}
Si prova per vari valori di W, contando per ogni valore quante tuple vengono generate -> grafico -> prendiamo la W associata al suo knee.
\\
Generico:
\\
\\Mercury: DimWindow = 150
\\BlueGene: DimWindow = 150
\section{Data Analysis}
Analisi statistica dei dati ottenuti.

