\chapter{Reliability}
\section{Esercizio 1}
Calcolare \textbf{Reliability} e \textbf{MTTF} per il sistema il cui Reilability Block Diagram è rappresentato nell'immagine sottostante. Assumere che tutti i componenti sono identici e falliscono randomicamente con tasso di fallimento \textit{$\lambda$}.
\begin{figure}[H]
	\centering
	\includegraphics[width=1\textwidth]{img/hw5/es1.png}
	\caption{\textit{Reliability Block Diagram Es.1}}
\end{figure}
\subsection{Svolgimento}
Dalla traccia si evince che la \textit{reliability} di un singolo blocco del sistema in questione ha andamento esponenziale ed è:
\begin{equation*}
	R_i = e^{-\lambda*t}
\end{equation*}
\subsubsection{Success Diagram}
Non è ancora possibile riconoscere una serie da un parallelo in questo caso, quindi una prima cosa da fare è ricavare un \textit{success diagram}, ovvero un sistema composto da un parallelo di serie. Le serie sono anche dette \textit{success path} e coincidono con tutti i percorsi possibili dall'ingresso all'uscita del sistema. 
\begin{figure}[H]
	\centering
	\includegraphics[width=0.7\textwidth]{img/hw5/success_diag.png}
	\caption{\textit{Success Diagram Es.1}}
\end{figure}
\begin{equation*}
	Rsys \leq 1 - \prod_{i=1}^{N}(1 - Rpath_i)
\end{equation*}
Con $N$ pari al numero di serie del diagramma, e $Rpath_i$ la reliability del path i-esimo.
\\Nel caso specifico dell'esercizio:
\begin{equation*}
	Rsys \leq 1 - [(1 - R_A*R_B*R_C)*(1-R_A*R_B*R_F)*(1-R_D*R_E*R_F)*(1-R_D*R_B*R_C)*(1-R_D*R_B*R_F)]
\end{equation*}
In questo modo è possibile ricavare agevolmente una reliability che sarà un upper bound per quella effettiva del sistema. Difatti i vari path non sono tra di loro indipendenti, dato che il fallimento di un blocco potrebbe interessare più di una serie.
\begin{equation*}
	Rsys \leq 1 - (1 - e^{-3\lambda*t})^{5}
\end{equation*}
\subsubsection{Conditioning}
La tecnica del \textbf{conditioning} consente di ricavare la reliability di un sistema facendo uso della formula del teorema di Bayes:
\begin{equation*}
	P(A) = \sum_{i = 1}^{N}P(A/B_i)P(B_i)
\end{equation*}
In poche parole, dato il sistema, viene supposto che uno dei suoi componenti $R_m$ (quello più critico per l'analisi) sia fallito o meno. Otteniamo quindi due versioni del diagramma iniziale:
\begin{enumerate}
	\item \textit{con componente selezionato up (circuito chiuso)}.
	\item \textit{con componente selezionato down (circuito aperto)}.
\end{enumerate}
\begin{equation*}
	Rsys = Rsys_1 + Rsys_2 = R_m*P(sys\,works|m\,up) + (1 - R_m)*P(sys\,works|m\,down)
\end{equation*}
\subsubsection{Conditioning sul blocco E}
\begin{figure}[H]
	\centering
	\includegraphics[width=0.7\textwidth]{img/hw5/e_down.png}
	\caption{\textit{Diagramma con Blocco E down Es.1}}
\end{figure}
Il sistema è composto da 2 paralleli in serie con il blocco B. La reliability totale può essere calcolata agevolmente sfruttando le formule di blocchi in serie e in parallelo.

\begin{equation*}
	Rsys_2 = (1-R_E)*P(sys\,works|E\,down)
\end{equation*}
\begin{equation*}
	\begin{split}
		Rsys_2 &= (1-R_E)*\{[1-(1-R_A)*(1-R_D)]*R_B*[1-(1-R_C)*(1-R_F)]\} \\
		&= (1-e^{-\lambda t})*[1-(1-e^{-\lambda t})^{2}]^{2}*e^{-\lambda t} \\
		&= (1-e^{-\lambda t})*[1-(1+e^{-2\lambda t}-2e^{-\lambda t})]^{2}*e^{-\lambda t} \\
		&= (1-e^{-\lambda t})*[-e^{-2\lambda t}+2e^{-\lambda t}]^{2}*e^{-\lambda t} \\
		&= (1-e^{-\lambda t})*[e^{-4\lambda t}+4e^{-2\lambda t}-4e^{-3\lambda t}]*e^{-\lambda t} \\
		&=(1-e^{-\lambda t})*[e^{-5\lambda t}+4e^{-3\lambda t}-4e^{-4\lambda t}] \\
		&= [e^{-5\lambda t}+4e^{-3\lambda t}-4e^{-4\lambda t}] - [e^{-6\lambda t}+4e^{-4\lambda t}-4e^{-5\lambda t}] \\
		&= 4e^{-3\lambda t}-8e^{-4\lambda t}+5e^{-5\lambda t}-e^{-6\lambda t}
	\end{split}
\end{equation*}
\begin{figure}[H]
	\centering
	\includegraphics[width=0.7\textwidth]{img/hw5/e_up.png}
	\caption{\textit{Diagramma con Blocco E up Es.1}}
\end{figure}
\begin{equation*}
	Rsys_1 = R_E*P(sys\,works|E\,up)
\end{equation*}
Tuttavia il sistema così ottenuto deve ancora essere condizionato, visto che non essendo combinazione di serie e/o paralleli non è possibile ricavare immediatamente la probabilità condizionata.
\subsubsection{Conditioning sul blocco B - E up}
\begin{figure}[H]
	\centering
	\includegraphics[width=0.7\textwidth]{img/hw5/b_down.png}
	\caption{\textit{Blocco E up e Blocco B down Es.1}}
\end{figure}
\begin{figure}[H]
	\centering
	\includegraphics[width=0.7\textwidth]{img/hw5/b_up.png}
	\caption{\textit{Blocco E up e Blocco B up Es.1}}
\end{figure}
\begin{equation*}
	P(sys\,works|E\,up) = P(sys\,works|B\,down)*P(B\,down)+P(sys\,works|B\,up)*P(B\,up) =
\end{equation*}

\begin{equation*}
	\begin{split}
	P(sys\,works|E\,up) &= (1-R_B)*R_D*R_F + R_B*[1-(1-R_A)*(1-R_D)]*[1-(1-R_C)*(1-R_F)]  \\
	&= (1-e^{-\lambda t})*e^{-2\lambda t} + e^{-\lambda t}*[1-(1-e^{-\lambda t})^{2}]^{2} \\
	&= e^{-2\lambda t} - e^{-3\lambda t} + e^{-\lambda t}*[e^{-4\lambda t}+4e^{-2\lambda t}-4e^{-3\lambda t}] \\
	&= e^{-2\lambda t} - e^{-3\lambda t} + e^{-5\lambda t}+4e^{-3\lambda t}-4e^{-4\lambda t} \\
	&= e^{-2\lambda t} + 3e^{-3\lambda t} -4e^{-4\lambda t} +e^{-5\lambda t} 
	\end{split}
\end{equation*}
Dunque otteniamo che :
\begin{equation*}
	\begin{split}
		Rsys_1 &= e^{-\lambda t}*[e^{-2\lambda t} + 3e^{-3\lambda t} -4e^{-4\lambda t} +e^{-5\lambda t}] \\
		&=  e^{-3\lambda t} + 3e^{-4\lambda t} -4e^{-5\lambda t} +e^{-6\lambda t}
	\end{split}
\end{equation*}
Quindi la \textbf{reliability totale} del sistema sarà:
\begin{equation*}
	\begin{split}
		Rsys &= Rsys_1 + Rsys_2 \\
		&= 4e^{-3\lambda t}-8e^{-4\lambda t}+5e^{-5\lambda t}-e^{-6\lambda t} + e^{-3\lambda t} + 3e^{-4\lambda t} -4e^{-5\lambda t} +e^{-6\lambda t} \\
		&= 5e^{-3\lambda t}-5e^{-4\lambda t}+ e^{-5\lambda t}
	\end{split}
\end{equation*}
Mentre il \textbf{MTTF - Mean Time To Failure} è:
\begin{equation*}
	MTTF = \int_{0}^{\infty} Rsys(t)dt = \frac{5}{3\lambda}-\frac{5}{4\lambda}+\frac{1}{5\lambda}
\end{equation*}


\section{Esercizio 2}
Confrontare i due schemi diversi di uno stesso sistema che sfrutta la ridondanza. Supponendo che il sistema ha bisogno di \textit{s} componenti identici in serie per le proprie operazioni. Inoltre siano dati $m \times s$ componenti totali. 
\begin{enumerate}
	\item Data la reliability di un singolo componente pari a $R$, ricavare l'espressione della reliability delle due configurazioni.
	\\Per $m=3$ e $s=4$, confrontare le due espressioni in funzione del \textit{mission time} $t$.
	\item Dati i due schemi nella figura sottostante, quale avrà una reliability maggiore? Modificare lo schema che ha la reliability minore in modo da raggiungere la stessa reliability dell'altro.
\end{enumerate}
Sia MTTF del singolo componente pari a \textit{100 ore}.
\begin{figure}[H]
	\label{es2}
	\centering
	\includegraphics[width=0.7\textwidth]{img/hw5/es2.png}
	\caption{\textit{Reliability Block Diagrams Es.2}}
\end{figure}
\subsection{Svolgimento}
In entrambi i sistemi si possono riconoscere facilmente le parti che hanno componenti in serie e in parallelo. Non c'è bisogno di utilizzare tecniche come \textit{Legge di Bayes} o \textit{Teorema dell'Upperbound} per ricavare la legge della reliability complessiva.
\subsubsection{Sistema A}
Per "Sistema A" si intende il sistema in alto che compare in Fig.\ref{es2}.
\\Per $m=3$ e $s=4$ esso ha 4 componenti in serie, disposti all'interno di 3 blocchi in parallelo. Quindi:
\begin{equation*}
	\begin{split}
			&R_{serie} = \prod_{i=1}^{4} R_i = R^4 \\
			&R_{parallelo} = 1 - \prod_{j=1}^{3}(1-R_{serie}) = (1-R_{serie})^3 \\
			&R_A = 1-(1-R^4)^3
	\end{split}
\end{equation*}
\subsubsection{Sistema B}
Per "Sistema B" si intende il sistema in basso che compare in Fig.\ref{es2}.
\\Per $m=3$ e $s=4$ esso ha 4 blocchi in serie, ognuno dei quali contine 3 componenti in parallelo. Quindi:
\begin{equation*}
	\begin{split}
		&R_{parallelo} = 1 - \prod_{j=1}^{3}(1-R_{j}) = 1-(1-R)^3 \\
		&R_{serie} = \prod_{i=1}^{4}R_{parallelo} = R_{parallelo}^4 \\
		&R_B = [1-(1-R)^3]^4
	\end{split}
\end{equation*}
\vspace{0.5cm}
\\Dato che le due reliability $R_A$ e $R_B$ sono in funzione del \textit{mission time} $t$, per confrontarle si può valutare un grafico per vari valori del tempo $t$.
Sapendo che:
\begin{equation*}
	MTTF = 100h \quad \Rightarrow \quad \lambda = \dfrac{1}{100h}
\end{equation*}
Si ha:
\begin{equation}
	\begin{split}
			&R_A(t) = 1 - (1 - e^{-4\lambda t})^3 \\
			&R_B(t) = [1-(1-e^{-\lambda t})^3]^4
	\end{split}
\end{equation}
Con un semplice script MATLAB:
\begin{minted}[framesep = 1mm,
	fontsize = \footnotesize,
	breaklines,
	]{MATLAB}
MTTF = 100; % [h]
lambda = 1/MTTF;
m = 3; s = 4;
t = 0:1:200;
R = exp(-lambda*t);  %PDF

R1 = 1-(1-R.^s).^(m);   % Reilability del Sistema 1
R2 = (1-(1-R).^m).^s;   % Reilability del Sistema 2

figure;
plot(t,R1); hold on; plot(t,R2);
legend("Sistema Serie (A)", "Sistema Parallelo (B)");
grid;
\end{minted}
Fornendo come risultato il seguente grafico.
\begin{figure}[H]
	\centering
	\includegraphics[width=\textwidth]{img/hw5/es2_grafico1.png}
	\caption{\textit{Confronto tra le reliability}}
\end{figure}
Come si può notare il "Sistema B" ha una reliability sempre maggiore rispetto al "Sistema A", per ogni \textit{mission time}. In realtà la stessa considerazione poteva essere fatta anche senza calcolare le due espressioni $R_A$ e $R_B$, poiché il secondo sistema ha un numero di \textit{success path} pari a $m^s$ mentre il primo sistema pari a $m$. Di conseguenza essendoci più percorsi alternativi, l'affidabilità è sicuramente maggiore.

\vspace{0.5cm}
Il "Sistema A" può essere opportunamente modificato per raggiungere la stessa reliability del "Sistema B" a parità di \textit{mission time}. Dato che il parametro $s$ non può essere cambiato (vincolo definito dalla traccia) l'unico parametro che può variare è $m$, ovvero il numero di blocchi in parallelo. 
\\$R_A$ e $R_B$ non sono la stessa funzione, quindi non esiste nessun valore di $m$ per cui le due funzioni si sovrappongono. Per raggiungere la stessa reliability allora bisogna prima fissare il \textit{mission time} e poi calcolare l'$m$ necessario. \\Per far ciò, in linea generale, posto $k$ come il numero di blocchi-serie da mettere in parallelo deve accadere:
\begin{equation}
	\begin{split}
		&1-(1-R(t)^s)^k = R_B(t) \\
		&1-R_B(t) = (1-R(t)^s)^k \\
		&k = \dfrac{ln(1-R_B(t))}{ln(1-R(t)^s)} 
	\end{split}
\end{equation}
Quindi $k$ dipende dal \textit{mission time}. Effettuando un grafico del valore di $k$ per vari istanti di tempo $t$, nel caso in esame con $m=3$ e $s=4$:
\begin{figure}[H]
	\centering
	\includegraphics[width=\textwidth]{img/hw5/es2_grafico2.png}
	\caption{\textit{Numero di Blocchi-Serie in funzione del Mission Time}}
\end{figure}
Se si volesse avere quindi la stessa reliability per i due sistema con $t=100$ (ore) allora $k\approx20$, servendosi di circa $20\times 4 = 80$ componenti.
\begin{figure}[H]
	\centering
	\includegraphics[width=\textwidth]{img/hw5/es2_grafico3.png}
	\caption{\textit{Confronto tra le reilability dei due sistemi}}
\end{figure}

\section{Esercizio 3}
L'architettura di una rete di computer in un sistema bancario e la seguente. L'architettura è chiamata \textbf{Skip-Ring Network} ed è progettata per consentire ai componenti di comunicare anche in presenza di nodi falliti.
\\Per esempio, se il \textit{Nodo 1} fallisce, il \textit{Nodo 8} può oltrepassare il nodo fallito attraverso un link alternativo, potendo raggiungere il \textit{Nodo 2}. 
\\Assumendo che tutti i link sono perfetti e ogni nodo ha una reilability $R_m$, ricavare l'espressione della reilability della rete.
\\Se $R_m$ ha una legge di fallimento esponenziale e il \textit{failure rate} di ogni nodo è di 0.005 fallimenti per ora, determinare la reilability del sistema dopo un periodo di 48 ore.
\begin{figure}[H]
	\centering
\includegraphics[width=0.7\textwidth]{img/hw5/es3_traccia.png}
\caption{\textit{Traccia}}
\end{figure}

\subsection{Svolgimento}
Il sistema funziona anche in presenza di nodi guasti, purché non si guastino nodi consecutivi. In tale scenario infatti l'anello viene interrotto e il sistema smette di funzionare. Sulla base di questo si può notare che il sistema continua a funzionare anche se falliscono 4 nodi, purché non siano consecutivi.
\\Per ricavare la legge della reilability ci si può ispirare alla legge per un sistema generale \textit{M-out-of-N} in cui $M$ sono il numero di nodi che devono funzionare, su $N$ affinché il sistema continua a funzionare.
\\In questo caso $M=4$ e $N=8$, a causa del fatto che se falliscono più di 4 nodi, automaticamente almeno due nodi falliti saranno consecutivi, interrompendo l'anello.
\begin{equation*}
	R_{sys} = \sum_{i=0}^{N-M}g(i)R_m^{N-i}(1-R_m)^i
\end{equation*}
in cui $i$ indica il numero di nodi falliti e $g(i)$ indica il numero di permutazioni di quel fallimento.
\\Nel caso specifico:
\begin{equation*}
	R_{sys} = \sum_{i=0}^{4}g(i)R_m^{N-i}(1-R_m)^i
\end{equation*}
\begin{itemize}
	\item i=0. Nessun nodo è guasto.
	\begin{equation*}
		g(0) = 1 
	\end{equation*}
	Poiché è un evento che non dipende dal numero dei nodi. La reilability al'iterazione 0 vale:
	\begin{equation*}
		R_{sys}[0] = R_m^8
	\end{equation*}
	\item i=1. Un solo nodo è guasto. Il sistema continua a funzionare normalmente. Al massimo si possono guastare $N$ nodi.
	\begin{equation*}
		g(1) = 8
	\end{equation*}
	La reilability vale:
	\begin{equation*}
	R_{sys}[1] = 8 R_m^7 (1-R_m)
	\end{equation*}	
	\item i=2. Due nodi sono guasti. In questo caso bisogna prestare attenzione. Il sistema continua a funzionare se i nodi guasti non sono adiacenti. Il numero di permutazioni vale come tutte le possibili permutazioni dei due nodi, sottratto il numero di combinazioni in cui i due nodi sono adiacenti:
	\begin{equation*}
		g(2) = \begin{pmatrix}
			8 \\ 2
		\end{pmatrix} - 8 = 20
	\end{equation*}
	Quindi:
	\begin{equation*}
		R_{sys}[2] = 20 R_m^6 (1-R_m)^2
	\end{equation*}	
	\item i=3. Se falliscono 3 nodi, come prima, bisogna stare attenti a considerare solo le combinazioni per cui non ci siano due nodi guasti consecutivi. Al totale delle combinazioni devono essere sottratte:
	\begin{enumerate}
		\item Combinazioni in cui 3 nodi sono adiacenti, come nel caso precedente, ovvero 8 combinazioni.
		\item Combinazioni in cui 2 nodi sono adiacenti e il terzo non è adiacente. Fissati due nodi adiacenti, il terzo non è adiacente solo in 4 casi. \\Ad esempio se il \textit{Nodo 2} e il \textit{Nodo 3} sono guasti, il terzo nodo deve essere tra \textit{Nodo 5}, \textit{Nodo 6}, \textit{Nodo 7}, \textit{Nodo 8}. Dato che può capitare per 8 volte, allora si hanno $8*4=32$ combinazioni.
	\end{enumerate}
	\begin{equation*}
		g(3) = \begin{pmatrix}
			8 \\ 3
		\end{pmatrix} - 8 - 32 = 16
	\end{equation*}
	\begin{equation*}
		R_{sys}[3] = 16 R_m^5 (1-R_m)^3
	\end{equation*}	
	\item i=4. Quattro nodi guasti. Questo è un caso limite, poiché i quattro nodi guasti devono alternarsi tra loro, altrimenti si avranno necessariamente due nodi guasti adiacenti. Quindi esistono solo due combinazioni:
	\begin{enumerate}
		\item Nodi: 1-3-5-7
		\item Nodi: 2-4-6-8
	\end{enumerate}
	\begin{equation*}
	g(4) = 2
	\end{equation*}
	Per cui:
	\begin{equation*}
		R_{sys}[4] = 2 R_m^4 (1-R_m)^4
	\end{equation*}
\end{itemize}
Avendo calcolato il vettore $g(i)$ la reilability complessiva vale:
\begin{equation*}
	R_{sys} = R_m^8 + 8 R_m^7 (1-R_m) + 20 R_m^6 (1-R_m)^2 + 16 R_m^5 (1-R_m)^3 + 2 R_m^4 (1-R_m)^4
\end{equation*}
Dato che si conosce il failure rate, e ogni componente ha una legge di reilability esponenziale:
\begin{equation*}
	R_m = e^{-\lambda t} = e^{-0.005 t}
\end{equation*}
Per cui il grafico complessivo di $R_{sys}$ è il seguente.
\begin{figure}[H]
	\centering
	\includegraphics[width=\textwidth]{img/hw5/es3_grafico.png}
	\caption{\textit{Reilabilty del sistema}}
\end{figure}
Inoltre dopo 48h di lavoro, il sistema ha una reilability pari a:
\begin{equation*}
	R_sys(48) \approx 0,7289
\end{equation*}

\section{Esercizio 4}
Confrontare la reilability dei seguenti sistemi, assumendo un \textit{Mean Time to Failure} esponenziale, con i seguenti valori:
\begin{itemize}
	\item \textit{$MTTF_A$ = 1000h}
	\item \textit{$MTTF_B$ = 9000h}
	\item \textit{$MTTF_C$ = 2000h}
\end{itemize}
\begin{figure}[H]
	\centering
	\includegraphics[width=0.8\textwidth]{img/hw5/es4_traccia.png}
	\caption{\textit{Sistemi da confrontare}}
\end{figure}
\subsection{Svolgimento}
Innanzitutto sono state calcolati tassi di fallimento $\lambda$:
\begin{equation*}
	\begin{split}
		&\lambda_A = \frac{1}{MTTF_A} = \frac{1}{1000h}\\
		&\lambda_B = \frac{1}{MTTF_B} = \frac{1}{9000h}\\
		&\lambda_C = \frac{1}{MTTF_C} = \frac{1}{2000h}
	\end{split}
\end{equation*}
e di conseguenza le reliability relative ad ogni singolo componente:
\begin{equation*}
	\begin{split}
		&R_A(t) = e^{-\lambda_A t} = e^{-\frac{1}{1000h} t}\\
		&R_B(t) = e^{-\lambda_B t} = e^{-\frac{1}{9000h} t}\\
		&R_C(t) = e^{-\lambda_C t} = e^{-\frac{1}{2000h} t}
	\end{split}
\end{equation*}
\subsubsection{Confronto 1}
Confrontiamo il parallelo tra le serie (A e B) e (A e C), con la serie tra il blocco A e il parallelo tra i blocco B e C.
Reliability \textbf{sistema 1}:
\begin{equation*}
	\begin{split}
		&R11(t) = 1-(1-R_A*R_B)*(1-R_A*R_C) = \\
		&1-(1-e^{-\frac{1}{1000h} t}*e^{-\frac{1}{9000h} t})*(1-e^{-\frac{1}{1000h} t}*e^{-\frac{1}{2000h} t}) = 1 -(1-e^{\frac{1}{900h} t})(1-e^{-\frac{3}{2000h}t}) = \\
		&e^{-\frac{3}{2000h}t}+e^{-\frac{1}{900h}t}-e^{-\frac{47}{18000h}t}
	\end{split}
\end{equation*}
Reilability \textbf{sistema 2}:
\begin{equation*}
	\begin{split}
		&R12(t) = R_A * [1-(1-R_B)*(1-R_C)] = e^{-\frac{1}{1000h} t}*[1-(1-e^{-\frac{1}{9000h} t})*(1-e^{-\frac{1}{2000h} t})] = \\
		& e^{-\frac{3}{2000h}t}+e^{-\frac{1}{900h}t}-e^{-\frac{29}{18000h}t}
	\end{split}
\end{equation*}
\begin{figure}[H]
	\centering
	\includegraphics[width=\textwidth]{img/hw5/es4_1.png}
	\caption{\textit{Primo confronto}}
\end{figure}
La reliability del primo sistema (in blu nel grafico) risulta, con evidenza, maggiore rispetto a quella del secondo, fino a valori di t pari a circa 4000h.
\subsubsection{Confronto 2}
Nel secondo punto confrontiamo la reliability del blocco A messo in serie ad un parallelo tra A e B, con quella del singolo blocco A. 
\\
Reliability \textbf{sistema 1}:
\begin{equation*}
	\begin{split}
		&R21= R_A*[1-(1-R_A)*(1-R_B)] =e^{-\frac{1}{1000h} t}*[1-(1-e^{-\frac{1}{1000h} t})*(1-e^{-\frac{1}{9000h} t})] = \\
		&e^{-\frac{1}{900h}t}+e^{-\frac{1}{500h}t}+e^{-\frac{19}{9000h}t}
	\end{split}
\end{equation*}
Reliability \textbf{sistema 2}:
\begin{equation*}
	R22 = R_A = e^{-\frac{1}{1000h}}
\end{equation*}
\begin{figure}[H]
	\centering
	\includegraphics[width=\textwidth]{img/hw5/es4_2.png}
	\caption{\textit{Secondo confronto}}
\end{figure}
La reliability del blocco A, per valori di mission time compresi tra 1000h e 5000h circa, risulta essere maggiore di quella del sistema 1.
\subsubsection{Confronto 3}
In questo caso viene confrontato un primo sistema composto dai blocchi A e B in serie tra di loro e a sua volta in serie con il loro parallelo, ed un secondo sistema formato dalla semplice serie tra A e B.
\\
Reliability \textbf{sistema 1}:
\begin{equation*}
	\begin{split}
		&R31 = R21*R_B = e^{-\frac{1}{1000h} t}*e^{-\frac{1}{9000h} t}*[1-(1-e^{-\frac{1}{1000h} t})*(1-e^{-\frac{1}{9000h} t})] = \\
		&e^{-\frac{11}{9000h}t}+e^{-\frac{19}{9000h}t}+e^{-\frac{2}{900h}t}
	\end{split}
\end{equation*}
Reliability \textbf{sistema 2}:
\begin{equation*}
	\begin{split}
		&R32 = R_A*R_B = e^{-\frac{1}{1000h} t}*e^{-\frac{1}{9000h} t} = e^{-\frac{1}{900h}t}
	\end{split}
\end{equation*}
\begin{figure}[H]
	\centering
	\includegraphics[width=\textwidth]{img/hw5/es4_3.png}
	\caption{\textit{Terzo confronto}}
\end{figure}
Nel primo sistema (in blu), la serie con il parallelo implica una riduzione della reilability per mission time compresi tra 1000h e 5000h circa.
\subsubsection{Confronto 4}
Infine viene confrontato il parallelo tra il blocco A e la serie tra A e B, con il singolo blocco A.
\\
Reliability \textbf{sistema 1}:
\begin{equation*}
	\begin{split}
		&R41 = 1-(1-R_A)*(1-R_A*R_B) = 1-(1-e^{-\frac{1}{1000h} t})*(1-e^{-\frac{1}{1000h} t}*e^{-\frac{1}{9000h} t}) = \\
		&e^{-\frac{1}{900h}t}+e^{-\frac{1}{1000h}t}-e^{-\frac{19}{9000h}t}
	\end{split}
\end{equation*}
Reliability \textbf{sistema 2}:
\begin{equation*}
	R42 = R_A = e^{-\frac{1}{1000h}}
\end{equation*}
\begin{figure}[H]
	\centering
	\includegraphics[width=\textwidth]{img/hw5/es4_4.png}
	\caption{\textit{Quarto confronto}}
\end{figure}
La reliability del primo sistema risulta incrementata rispetto a quella del blocco A, come conseguenza del parallelo tra i blocchi.


\section{Esercizio 5}
Il sistema mostrato nella figura sottostante è un sistema di elaborazione per un elicottero. Esso ha due processori e due terminali ridondanti. Sono utilizzati due bus, anch'essi dual-redundant. La parte interessante del sistema è il "navigation equipment". Il velivolo può essere completamente navigato utilizzando l' \textit{Inertial Navigation System (INS)}. Se l'INS fallisce, il velivolo può essere navigato utilizzando la combinazione del \textit{Doppler} con l' \textit{Altitude Heading and Reference System (AHRS)}. Il sistema contiene 3 unità AHRS, delle quali solo una è necessaria. Questo è un esempio di ridondanza funzionale dove i dati provenienti dall'AHRS e dal Doppler possono essere utilizzati per rimpiazzare l'INS, se esso fallisce. A causa di altri sensori e strumenti, entrambi i bus sono necessari per il funzionamento del sistema a prescindere dalla modalità di navigazione.
\begin{figure}[H]
	\centering
	\includegraphics[width=0.7\textwidth]{img/hw5/es5_traccia.png}
	\caption{\textit{Processing system for a helicopter}}
\end{figure}
\begin{enumerate}[ A) ]
	\item Disegnare il Reilability Block Diagram del sistema
	\item Disegnare il Fault Tree del sistema e analizzare il \textit{minimal cutset}
	\item Calcolare la reilability per un'ora di volo utilizzando gli MTTF nella tabella seguente. Assumere che la legge di fallimento è esponenziale e che la \textit{fault coverage} è perfetta
	\item Ripetere il punto precedente, ma questa volta, considerare un fattore per la \textit{fault detection} e la riconfigurazione delle \textit{processing units}. Usando i dati della stessa tabella, determinare il valore approssimato di \textit{fault coverage} che è richiesto per ottenere una reilability (al termine di un'ora) di 0.99999.
\end{enumerate}
\begin{table}
\centering
\begin{tabular}{|l|l|}
	\hline
	Equipment & MTTF (hr) \\
	\hline
	\hline
	Processing Unit & 10000 \\
	Remote Terminal & 4500 \\
	AHRS & 2000 \\
	INS & 2000 \\ 
	Doppler & 500 \\
	Bus & 60000 \\
	\hline
\end{tabular}
\end{table}
\subsection{Punto A}
Possiamo suddividere il sistema in cinque "macro-blocchi":
\begin{enumerate}
	\item \textbf{Bus A}.
	\item \textbf{Bus B}.
	\item \textbf{Sistema di navigazione}.
	\item \textbf{Processing unit}.
	\item \textbf{Remote terminal}.
\end{enumerate}
Se almeno uno di essi fallisce, il sistema intero fallisce. Ogni macro-blocco, (eccetto il sistema di navigazione per il quale è stata pensata una particolare ridondanza funzionale già descritta nella traccia), è dual-redundant. Inoltre, ciascuna delle due \textit{Processing Unit} è connessa ad entrambi i bus, in modo da avere dual-redoundance sul singolo bus. 
\\
\\
Partendo da queste informazioni è stato realizzato il seguente \textit{Reliability Block Diagram}:
\begin{figure}[H]
	\label{RBD}
	\centering
	\includegraphics[width=\textwidth]{img/hw5/es5_RBD.png}
	\caption{\textit{Reliability Block Diagram}}
\end{figure}
\newpage

\subsection{Punto B}
Partendo dal Reliability Block Diagram è possibile ricavare il rispettivo Fault Tree, considerando che il sistema fallisce se il \textit{Top Level Event} ha valore logico 1.
\\In particolare lo serie saranno sostituite con delle porte OR, mentre i paralleli con porte END. 
\begin{figure}[H]
	\centering
	\includegraphics[width=\textwidth]{img/hw5/es5_FT.png}
	\caption{\textit{Fault Tree}}
\end{figure}
Espressione logica del \textit{Fault Tree}:
\begin{equation*}
	\begin{split}
		&\textbf{TLE} = (A1*A2)+(B1*B2)+{INS_A*[DOP_A+(AHRS_A*AHRS_B*AHRS_B)]}+ \\
		&(P_AB*P_AB)+(T_A*T_B) \\
		&=(A1*A2)+(B1*B2)+(INS_A*DOP_A)+(INS_A*AHRS_A*AHRS_B*AHRS_B)+ \\
		&(P_AB*P_AB)+(T_A*T_B)
	\end{split}
\end{equation*}
I \textit{minimal cutsets} sono per definizione il minimo set di eventi di base che portano al fallimento del sistema (TLE = 1). In questo caso sono stati ricavati dall'espressione logica del Fault Tree, la quale è stata espressa mediante somme di prodotti. Tali prodotti rappresentano proprio i minimal cutsets.
\\
Dunque il sistema fallisce, se falliscono congiuntamente:
\begin{itemize}
	\item \textit{Le due repliche del bus A (A1 e A2)},
	\item \textit{Le due repliche del bus B (B1 e B2)},
	\item \textit{L'INS e il Doppler}, in quanto non sarebbe possibile replicare il sistema di navigazione,
	\item \textit{L'INS e i tre AHRS}, per lo stesso motivo di prima,
	\item \textit{Le due repliche della Processing Unit},
	\item \textit{Le due repliche del Remote Terminal}.
\end{itemize}
Ovviamente queste sono solo le minime combinazioni di eventi che portano al fallimento del sistema, ma non le uniche.

\subsection{Punto C}
Per ogni blocco la legge descrive la reilability è di tipo esponenziale e vale:
\begin{equation*}
	R_c = e^{-\lambda_c t} \quad con \quad \lambda_c=\dfrac{1}{MTTF_c}
\end{equation*}
in cui $MTTF_c$ rappressenta il Mean Time To Failure del singolo componente $c$, descritto nella tabella.
\\A partire dalla Fig.\ref{RBD} si notano immediatamente componenti in serie e in parallelo. In particolare ci sono 5 blocchi in serie, per ognuno dei quali si può calcolare la reilability.
\subsubsection{Bus A}
Il primo blocco ha in parallelo le due componenti replicate del bus.
\begin{equation*}
	\begin{split}
		R_{BusA} &= 1-(1-R_A)^2 \\
		&= 1-(1-e^{-\frac{t}{60000}})^2 \\
		&= 2e^{-\frac{t}{60000}} - e^{-2\frac{t}{60000}}
	\end{split}
\end{equation*}
\subsubsection{Bus B}
Analogamente avviene per il bus B.
\begin{equation*}
	\begin{split}
		R_{BusB} &= 1-(1-R_B)^2 \\
		&= 2e^{-\frac{t}{60000}} - e^{-2\frac{t}{60000}}
	\end{split}
\end{equation*}
\subsubsection{Sistema di Navigazione}
Il ramo superiore del parallelo rappresenta l'alternativa quando il componente INS si guasta. La sua reilability vale:
\begin{equation*}
	\begin{split}
			R_{secondario} &= R_{DOP} * [1-(1-R_{AHRS})^3] \\
			&=e^{-\frac{t}{500}}[1-(1-e^{-\frac{t}{2000}})^3] \\
			&=e^{-\frac{t}{500}}[1-(1 - 3e^{-\frac{t}{2000}} + 3e^{-2\frac{t}{2000}} - e^{-3\frac{t}{2000}})] \\
			&=e^{-\frac{t}{500}}[3e^{-\frac{t}{2000}} - 3e^{-2\frac{t}{2000}} + e^{-3\frac{t}{2000}})] \\
			&= 3e^{-\frac{t}{400}} - 3e^{-3\frac{t}{1000}} + e^{-7\frac{t}{2000}}
	\end{split}
\end{equation*}
Il ramo inferiore rappresenta il componente principale del sistema di navigazione.
\begin{equation*}
	R_{principale} = R_{INS} = e^{-\frac{t}{2000}}
\end{equation*}
In definitiva:
\begin{equation*}
	\begin{split}
	 R_{NAV} &= 1-(1-R_{secondario})(1-R_{principale}) \\
	 &= 1 - \{1-R_{DOP}[1-(1-R_{AHRS})^3]\}\{1- R_{INS}\} \\
	 &= 1 - (1-3e^{-\frac{t}{400}} + 3e^{-3\frac{t}{1000}} - e^{-7\frac{t}{2000}})(1-e^{-\frac{t}{2000}})
	\end{split}
\end{equation*}

\subsubsection{Processing Unit}
L'unità di calcolo ha una struttura duale al bus, per cui:
\begin{equation*}
		\begin{split}
			R_{CPU} &= 1-(1-R_{pu})^2 \\
			&= 2e^{-\frac{t}{10000}} - e^{-2\frac{t}{10000}}
		\end{split}
\end{equation*}

\subsubsection{Remote Terminal}
Analogo discorso vale per il sistema dei terminali.
\begin{equation*}
		\begin{split}
			R_{TERM} &= 1-(1-R_term)^2 \\
			&= 2e^{-\frac{t}{4500}} - e^{-2\frac{t}{4500}}
		\end{split}
\end{equation*}
\vspace{0.5cm}
\\La reilability complessiva vale come il prodotto delle reilability dei singoli blocchi, essendo tutti collegati in serie.
\begin{equation}
	\label{reilability_totale}
	R_{sys} = R_{BusA} * R_{BusB} * R_{NAV} * R_{CPU} * R_{TERM} 
\end{equation}
\begin{figure}[H]
	\centering
	\includegraphics[width=\textwidth]{img/hw5/es5_reilability.png}
	\caption{\textit{Reilability del sistema con un intervallo di 8h}}
\end{figure}
In un'ora di volo la reilability vale:
\begin{equation*}
	R_{sys}(1) \approx 0.9999989413 
\end{equation*}


\subsection{Punto D}
Fin ora si è supposto che tutti i fallimenti di tutti i componenti possono essere rilevati con probabilità del 100\%, ma ciò non accade nella realtà. Ad esempio dal blocco della CPU si evince che se fallisce una CPU, si può utilizzare tranquillamente l'altra. Questo nell'ipotesi che il fallimento viene rilevato. Se il fallimento non viene rilevato allora si continua ad utilizzare una CPU guasta, portando ad un fallimento totale del sistema. 
\\Per ovviare a questo problema (nel caso della CPU) si aggiunge un fattore di \textit{fault coverage} $\textbf{c}$ che indica la probabilità che venga rilevato un fallimento.
\begin{equation}
	\label{reilability_CPU}
	R_{CPU} = 1-(1-R_{pu})^2 \quad\Rightarrow\quad R_{CPU} = R_{pu} + c(1-R_{pu})R_{pu}
\end{equation}
Essa rappresenta la reilability del parallelo tra le CPU con l'aggiunta della \textit{fault coverage}. Sostituendo la (\ref{reilability_CPU}) nella (\ref{reilability_totale}) si ottiene:
\begin{equation*}
	R_{sys} = R_{BusA} * R_{BusB} * R_{NAV} * [R_{pu} + c(1-R_{pu})R_{pu} ]* R_{TERM} 
\end{equation*}
Il valore della \textit{fault coverage} deve essere ricavato considerando che la reilability (in un'ora) vale 0.99999.
\begin{equation}
	c \geq \dfrac{\dfrac{R_{sys}}{R_{BusA}R_{BusB}R_{NAV}R_{TERM}} - R_{pu}}{R_{pu}(1-R_{pu})} = 0.91057
\end{equation}
