\chapter{FFDA}
Log-based FFDA di sistemi complessi a larga scala. I dati sono stati collezionati dai supercalcolatori Mercury e Blue-Gene. 
\\Breve descrizione dell'architettura dei due sistemi
\section{Analisi dei log}
Formato entries per entrambi.
\\RICORDA!! I dati sono stati già filtrati, ma comunque sono presenti ridondanze. Quindi necessitano una manipolazione.
\section{Manipolazione}
Fatta tramite tecniche di coalescenza, le quali consentono di raggruppare entries ricercando correlazioni sui dati. 
\\Coalescenza spaziale: raggruppiamo entries provenienti da + nodi diversi cercando di capire se si sono verificate nello stesso intervallo temporale specificato dalla finestra temporale W seleizonata.
\subsection{Scelta W}
Si prova per vari valori di W, contando per ogni valore quante tuple vengono generate -> grafico -> prendiamo la W associata al suo knee.
\\
Generico:
\\
\\Mercury: DimWindow = 150
\\BlueGene: DimWindow = 150
\subsection{Analisi Truncation}
Per qualche tupla d'esempio individuare se sono presenti troncamenti.
\subsection{Analisi Collisions}
Per qualche tupla d'esempio individuare collissioni.
\subsection{Domanda 1}
La stessa finestra di coalescenza può essere usata per diversi nodi (fare sia per Mercury che BG) e categorie d'errore (solo Mercury)?
\section{Reliability Modeling}
CDF e pdf (unreliability e reliability) degli interrarrivi (tempi a fallimento dei sistemi). Da fare prima sull'intero sistema (sia mercury che bg).
\subsection{Fitting di EmpRel}
\subsection{Domanda 2}
Relazione tra Rel dell'intero sistema e dei Nodi (magari usiamo gli stessi nodi della domanda 1).
\subsection{Domanda 3}
Esistono colli di bottiglia ? Nodi che contribuiscono maggiormente al numero totale di fallimenti. = Grafico nodi-nr fallimenti per nodo (vedere se fare tramite script matlab)
\subsection{Domanda 4}
Stessi nodi funzionali (es due nodi di login in mercury) esibiscono parametri di reliability simili?
\subsection{Domanda 5}
Esiste una relazione tra tipo di errore e nodo(mercury)?

