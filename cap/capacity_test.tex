\chapter{Capacity Test}
L'obiettivo del Capacity Test è quello di valutare le performance di un qualsiasi sistema quando è sottoposto a carichi di lavoro di diversa intensità, in modo da caratterizzare le sue prestazioni al limite (sotto condizioni di lavoro severe).
\\
Per realizzare queste valutazioni, sono stati collezionati, lato client, gli \textbf{high-level parameters}. Essi possono riferirsi alla richiesta (quando è stata fatta, chi l'ha fatta ecc..) o alla risposta (tempi di risposta, errori).
\\
Performance metrics : response time e throughput
\\
Knee capacity e usable capacity + immagine

\section{Experimental Setup}
Il sistema oggetto di studio è un \textit{Web Server Apache} installato sulla macchina virtuale guest, che funge da server.
\\
Tramite la modalità \textit{Host-only Network Adapter}, configurabile nelle impostazioni della macchina virtuale, è stato possibile far comunicare la macchina guest con quella host, che lo ospita. Su quest'ultima è stata installata l'applicazione Java \textit{JMeter}, che ha permesso l'analisi delle prestazioni complessive del Server, sottoponendolo a diversi tipi di carico.
\\
\\
In questa analisi è stato scelto di valutare le prestazioni in media del server, considerando solo richieste di tipo random (tipo = dimensione pagina, nel nostro esempio). 
\subsection{Server Setup}
Descrizione configurazione VM - non la so.
\\
Per creare un scenario reale, sul Server sono state caricate 5 pagine in formato testuale, di diversa dimensione:
\begin{itemize}
	\item \textbf{Small}, 50K
	\item \textbf{Small-Medium}, 100K
	\item \textbf{Medium}, 300K
	\item \textbf{Medium-Large}, 500K
	\item \textbf{Large}, 1M
\end{itemize}
Questi sono i file oggetto delle richieste realizzate da ipotetici client.
\subsection{Clients - JMeter Setup}
Innanzitutto è stato settato, nel \textit{ThreadGroup}, il numero di thread che JMeter usa per realizzare i test. Questa quantità rappresenta il numero di utenti "virtuali" che visitano il nostro server. Nel nostro esperimento sono stati previsti \textbf{50 threads}, un valore in linea con i suoi scopi (visto che il sito fa un po'schifo non è che può avere tantissimi utenti). In più, prevedendo dei test di durata pari a \textit{5 min}, sono stati impostati:
\begin{itemize}
	\item il \textbf{ramp-up period} - numero di secondi entro il quale deve essere attivato l'ultimo thread - a \textit{300 s}.
	\item il \textbf{thread lifetime} - durata massima di ogni thread - a \textit{300 s}.
	\item il \textbf{loop count} - numero di volte in cui un singolo thread viene attivato - a \textit{carico/50}.
\end{itemize}     
Al ThreadGroup sono stati aggiunti 5 \textit{HTTP Request Sampler}, uno per tipologia di richiesta da realizzare e nei quali sono stati specificati i path delle rispettive risorse sul server. Ad essi è stato integrato un \textit{Random Controller}, grazie al quale, quando un thread viene attivato, effettua solo una tra le cinque tipologie di richieste, selezionata in maniera randomica. Ancora una volta, aggiungendo variabilità alle nostre richieste, è stato possibile simulare una situazione realistica e, soprattutto, non predicibile.
\\
Attraverso il \textit{Constant Throughput Timer} è stato possibile impostare il carico da sottoporre al sistema, in termini di numero di richieste al minuto. Infine, il listener \textit{Simple Data Writer}, ci ha permesso di collezionare in un file, quei parametri di alto livello che sono d'interesse ai fini dell'esperimento.

\section{Esecuzione Capacity Test}
- specificare carico
- specificare come calcolare throughput
- differenza tra elapsed e latency e quale scegliamo come response time
- grafici e dati
